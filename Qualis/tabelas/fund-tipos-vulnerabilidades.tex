\begin{table}[!ht]
\centering
\fontsize{9.5pt}{10.25pt}\selectfont
\caption{Tipos de vulnerabilidades em contratos inteligentes}
\label{tab:sec-stud:almakhour-vulnerabilities}
\begin{tabular}{|l|l|} 
\hline
\textbf{Vulnerabilidades}                                                                              & \textbf{Mecanismo}                                                                                                                                                                                                                           \\ 
\hline
Reentrância                                                                                            & \begin{tabular}[c]{@{}l@{}}Chamada recursiva de uma função por meio de uma \\função \textit{fallback} \end{tabular}                                                                                                                          \\ 
\hline
\textit{Integer overflow/underflow}                                                                    & \begin{tabular}[c]{@{}l@{}}Um \textit{overflow/underflow} pode ocorrer quando são \\executadas operações de adição, subtração, ou \\armazenamento da entradas do usuário sobre \\variáveis inteiras com limitações de valor \end{tabular}    \\ 
\hline
Dependência da ordem da transação                                                                      & \begin{tabular}[c]{@{}l@{}}Ordem das transações inconsistente em relação \\ao momento da invocação \end{tabular}                                                                                                                             \\ 
\hline
\begin{tabular}[c]{@{}l@{}}Limite da pilha de chamadas\\ \textit{Call-stack depth limit} \end{tabular} & \begin{tabular}[c]{@{}l@{}}Quando é excedido o limite de chamadas ao\\~método de um contrato \end{tabular}                                                                                                                                   \\ 
\hline
Dependência do timestamp do bloco                                                                      & \begin{tabular}[c]{@{}l@{}}Ocorre quando um contrato que utiliza o \textit{timestamp} \\de um bloco como parte da condição para acionar\\~uma operação crítica (e.g. envio de ether) é \\explorado por um minerador malicioso \end{tabular}  \\ 
\hline
Autenticação por \textit{tx.origin}                                                                    & \begin{tabular}[c]{@{}l@{}}Ocorre quando um contrato utiliza \textit{tx.origin} para \\autenticação, a qual pode ser comprometida por \\um ataque \textit{phishing} \end{tabular}                                                            \\ 
\hline
DoS com operações limitadas                                                                            & \begin{tabular}[c]{@{}l@{}}Essa vulnerabilidade é resultante de programação \\imprópria com operações ilimitadas em um contrato \end{tabular}                                                                                                \\ 
\hline
Endereços curtos                                                                                       & A MVE não verifica a validade de endereços                                                                                                                                                                                                   \\ 
\hline
Contrato suicida (\textit{self-destruct})                                                              & \begin{tabular}[c]{@{}l@{}}Quando o contrato pode ser destruído por usuários \\não autorizados \end{tabular}                                                                                                                                 \\ 
\hline
Transferência não verificada e fracassada                                                              & Envio de ether sem checar possíveis condições                                                                                                                                                                                                \\ 
\hline
Saldo inseguro                                                                                         & \begin{tabular}[c]{@{}l@{}}Quando o saldo de um contrato é exposto devido a\\~utilização indevida do modificador \textit{public}, este pode \\ser roubado por um agente malicioso \end{tabular}                                              \\ 
\hline
Contrato guloso                                                                                        & \begin{tabular}[c]{@{}l@{}}Os fundos do contrato ou o saldo em ether são \\travados indefinidamente \end{tabular}                                                                                                                            \\ 
\hline
Contrato pródigo                                                                                       & \begin{tabular}[c]{@{}l@{}}Liberar fundos ou saldo em ether para usuários \\arbitrários \end{tabular}                                                                                                                                        \\ 
\hline
Exceção mal tratada                                                                                    & \begin{tabular}[c]{@{}l@{}}Quando uma exceção é disparada em um CI por \\meio da chamada de outro contrato e não é tratada \\devidamente pelo chamador \end{tabular}                                                                         \\ 
\hline
Gasto excessivo de \textit{gas}                                                                        & \begin{tabular}[c]{@{}l@{}}Consumo de \textit{gas} desnecessário na execução do \\código do contrato \end{tabular}                                                                                                                           \\
\hline
\end{tabular}
\fdireta{almakhour2020verification-survey}
\end{table}
% \FloatBarrier