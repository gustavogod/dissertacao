%Utilizar a tabela disponível em \cite{overview-ahmed-2019} para comparação entre os tipos de rede

O funcionamento de uma blockchain varia de acordo com a forma como se lida com questões como imutabilidade, permissões, velocidade de processamento de transações  e gerenciamento de participantes no processo de validação. Por isso, existem diferentes variações em blockchains de acordo com as necessidades das aplicações. Baseada nas características descritas por~\citeonline{overview-ahmed-2019} e~\citeonline{overview-consenso2017sankar}, esses tipos de blockchains podem ser classificados com segue:

\begin{itemize}
    \item \textbf{Blockchain pública:} A participação no processo de criação e validação de blocos, assim como o acesso ao histórico de transações, é aberto à todos os nós da rede. Todos os nós são livres para se juntar ou deixar a rede como desejarem. Os exemplos mais conhecidos de blockchain pública são a Bitcoin e a Ethereum, em que os mineradores validam as transações e recebem criptomoedas como recompensa pelo esforço em resolver o quebra-cabeça de \textit{hash};
    \item \textbf{Blockchain privada:} Não é publicamente acessível e possui uma estrutura centralizada. Regras para restrição de acesso e criação de blocos são estabelecidas, controladas e fiscalizadas por uma entidade central. Blockchains privadas são projetadas especialmente para empresas e companhias, como, por exemplo, a blockchain da Ripple~\cite{overview-schwartz2014ripple}. 
    \item \textbf{Blockchain de consórcio:} Nem todos os nós participantes possuem os mesmo direitos de validação de transações e participação no processo de consenso. Apenas o conjunto de nós servidores previamente selecionados podem controlar uma blockchain de consórcio e validar blocos, como acontece nas blockchains Hyperledger Fabric~\cite{overview-hyperledger2018androulaki} e Corda~\cite{brown2016corda}. 
\end{itemize}

Devido à sua natureza aberta para participação, uma blockchain pública é definida também como uma blockchain não permissiva. Já as blockchains privadas e de consórcio concedem direitos de acesso aos dados e validação de transações à nós específicos, e são, portanto, classificadas como blockchains permissivas. Há também casos como o da Ethereum, que, mesmo sendo uma blockchain utilizada como pública, também é possível configurar ambientes privados com controle de acesso. Mais detalhes sobre a rede Ethereum são tratados na Seção~\ref{tex:fund:ethereum}.